\documentclass[11pt,letterpaper]{article}
\usepackage[top=0.85in,left=1.30in,footskip=0.75in]{geometry}
\usepackage[titletoc,page]{appendix}
% Use adjustwidth environment to exceed column width (see example table in text)
\usepackage{changepage}
\usepackage[english]{babel}
\usepackage{booktabs}
\usepackage{siunitx}%Questo serve a caricare il pacchetto delle unità di misura del sistema internazionale%
\usepackage[utf8]{inputenc}
\usepackage{graphicx} 
\usepackage{url}
\usepackage{amsmath}
\usepackage{amssymb}
\usepackage{listings}


\usepackage{keyval}
\usepackage{xcolor}
\usepackage{caption}
\usepackage{tikz}
\usepackage{circuitikz}
\usepackage{authblk}
%\usepackage{hyperref}


\usepackage[lofdepth,lotdepth]{subfig}
% Remove comment for double spacing
%\usepackage{setspace} 
%\doublespacing
% Text layout
%\raggedright
\setlength{\parindent}{0.5cm}
\textwidth 5.25in 
\textheight 8.75in

\usepackage[aboveskip=1pt,labelfont=bf,labelsep=period,justification=raggedright,singlelinecheck=off]{caption}

% Use the PLoS provided BiBTeX style
\bibliographystyle{plos2009}

% Remove brackets from numbering in List of References
\makeatletter
\renewcommand{\@biblabel}[1]{\quad#1.}
\makeatother


\begin{document}
\vspace*{0.30in}

\begin{flushleft}
{\Large
\textbf\newline{\textbf{Master Equation for a chemical reaction}}
}
\newline
% Insert Author names, affiliations and corresponding author email.
\\
Lorenzo Maria Perrone\textsuperscript{1, *}
%Name2 Surname\textsuperscript{2,\textpilcrow},
%Name3 Surname\textsuperscript{2,\textcurrency a},
%Name4 Surname\textsuperscript{2,\ddag},
%Name5 Surname\textsuperscript{2,\ddag},
%Name6 Surname\textsuperscript{2,\Yinyang},
%Name7 Surname\textsuperscript{3,*,\Yinyang}
\\
\bf{1} Department of Physics, EPFL
\\
%\bf{2} Affiliation Dept/Program/Center, Institution Name, City, State, Country
%\\
%\bf{3} Affiliation Dept/Program/Center, Institution Name, City, State, Country
%\\

% Insert additional author notes using the symbols described below. Insert symbol callouts after author names as necessary.
% 
% Remove or comment out the author notes below if they aren't used.
%
% Primary Equal Contribution Note
%\Yinyang These authors contributed equally to this work.

% Additional Equal Contribution Note
%\ddag These authors also contributed equally to this work.

% Current address notes
%\textcurrency a Insert current address of first author with an address update
% \textcurrency b Insert current address of second author with an address update
% \textcurrency c Insert current address of third author with an address update

% Deceased author note
%\dag Deceased

% Group/Consortium Author Note
%\textpilcrow Insert Collaborative Author line here

* E-mail: lorenzo.perrone@epfl.ch
\end{flushleft}


\section{Exercise 4: Master equation for a chemical reaction}

We attach below the code used to compute the derivatives of the generating function (called \textbf{H} in what follows).\\
An array of values for $\alpha, \beta$ has been chosen in a range which seems compatible with the hypotheses of small fluxes. It has been noted that values of one parameter of the order 1E-4 or 1E-5 do not differ sensibly from the same computation in the 1E-3 order of magnitude. Moreover, the quantity we have defined as \textit{fluctuation} does not seem to be a function of the ratio of the parameters, but instead a function of each of them individually. This makes our problem bi-dimensional.\\

The definition of the fluctuation is:

\begin{equation}
\delta(\alpha, \beta) = \frac{\sqrt{\sigma_n^2}}{\langle n \rangle}
\end{equation}

The smallest fluctuation is found to be $\delta \sim 0.44$ and a realization for this value is, for example, $\beta = 9.82$, $\alpha = 0.03$.

\begin{lstlisting}
# -*- coding: utf-8 -*-
"""
Created on Wed Apr 12 15:05:56 2017

@author: LorenzoLMP
"""

from pylab import *  
from scipy import * 
from scipy import special
from scipy import misc



def H(z, alpha, beta):
    l = len(alpha);
    p1 = ((z+1)/2)**((1-2*beta)/2)
    p2 = special.iv(2*beta-1,sqrt(8*ones(l)*alpha*(z+1)))
    p3 = special.iv(2*beta-1,sqrt(16*ones(l)*alpha))
    return p1*p2/p3

N = 1000
alpha = logspace(-3,1,N)
beta = logspace(-3,1,N)    
    
mean_val = zeros((N,N))
variance = zeros((N,N))
for i in range(N):
    mean_val[i][:] = misc.derivative(H, 1, dx=1e-10, n=1, args=(alpha,beta[i]))
    variance[i][:] = sqrt(misc.derivative(H, 1, dx=1e-6, n=2, args=(alpha,beta[i])) 
    + mean_val[i][:] - mean_val[i][:]**2)

print(mean_val)


fluct = divide(variance, mean_val)

print('nonzero', nonzero(fluct[where( fluct<0.1 )]))
print(fluct[where( fluct<0.1 )])

\end{lstlisting}

\end{document}